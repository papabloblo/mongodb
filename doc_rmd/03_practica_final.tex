\documentclass[]{article}
\usepackage{lmodern}
\usepackage{amssymb,amsmath}
\usepackage{ifxetex,ifluatex}
\usepackage{fixltx2e} % provides \textsubscript
\ifnum 0\ifxetex 1\fi\ifluatex 1\fi=0 % if pdftex
  \usepackage[T1]{fontenc}
  \usepackage[utf8]{inputenc}
\else % if luatex or xelatex
  \ifxetex
    \usepackage{mathspec}
  \else
    \usepackage{fontspec}
  \fi
  \defaultfontfeatures{Ligatures=TeX,Scale=MatchLowercase}
\fi
% use upquote if available, for straight quotes in verbatim environments
\IfFileExists{upquote.sty}{\usepackage{upquote}}{}
% use microtype if available
\IfFileExists{microtype.sty}{%
\usepackage{microtype}
\UseMicrotypeSet[protrusion]{basicmath} % disable protrusion for tt fonts
}{}
\usepackage[margin=1in]{geometry}
\usepackage{hyperref}
\hypersetup{unicode=true,
            pdftitle={Práctica taller MongoDB},
            pdfauthor={Pablo Hidalgo},
            pdfborder={0 0 0},
            breaklinks=true}
\urlstyle{same}  % don't use monospace font for urls
\usepackage{graphicx,grffile}
\makeatletter
\def\maxwidth{\ifdim\Gin@nat@width>\linewidth\linewidth\else\Gin@nat@width\fi}
\def\maxheight{\ifdim\Gin@nat@height>\textheight\textheight\else\Gin@nat@height\fi}
\makeatother
% Scale images if necessary, so that they will not overflow the page
% margins by default, and it is still possible to overwrite the defaults
% using explicit options in \includegraphics[width, height, ...]{}
\setkeys{Gin}{width=\maxwidth,height=\maxheight,keepaspectratio}
\IfFileExists{parskip.sty}{%
\usepackage{parskip}
}{% else
\setlength{\parindent}{0pt}
\setlength{\parskip}{6pt plus 2pt minus 1pt}
}
\setlength{\emergencystretch}{3em}  % prevent overfull lines
\providecommand{\tightlist}{%
  \setlength{\itemsep}{0pt}\setlength{\parskip}{0pt}}
\setcounter{secnumdepth}{0}
% Redefines (sub)paragraphs to behave more like sections
\ifx\paragraph\undefined\else
\let\oldparagraph\paragraph
\renewcommand{\paragraph}[1]{\oldparagraph{#1}\mbox{}}
\fi
\ifx\subparagraph\undefined\else
\let\oldsubparagraph\subparagraph
\renewcommand{\subparagraph}[1]{\oldsubparagraph{#1}\mbox{}}
\fi

%%% Use protect on footnotes to avoid problems with footnotes in titles
\let\rmarkdownfootnote\footnote%
\def\footnote{\protect\rmarkdownfootnote}

%%% Change title format to be more compact
\usepackage{titling}

% Create subtitle command for use in maketitle
\providecommand{\subtitle}[1]{
  \posttitle{
    \begin{center}\large#1\end{center}
    }
}

\setlength{\droptitle}{-2em}

  \title{Práctica taller MongoDB}
    \pretitle{\vspace{\droptitle}\centering\huge}
  \posttitle{\par}
  \subtitle{Taller de NoSQL}
  \author{Pablo Hidalgo}
    \preauthor{\centering\large\emph}
  \postauthor{\par}
    \date{}
    \predate{}\postdate{}
  

\begin{document}
\maketitle

Entre las dos sesiones hemos visto cómo utilizar MongoDB para trabajar
contra una base de datos. Durante la primera sesión hicimos un
\emph{tour} rápido por alguno de los comandos más sencillos. Durante la
segunda sesión hicimos consultas más complicadas sobre el conjunto de
colecciones.

Para esta práctica vamos a continuar con las dos colecciones
\texttt{usage} y \texttt{stations} importadas en la última sesión.

\begin{quote}
\textbf{Pregunta 1:} extrae aquellos documentos de la colección
\texttt{usage} que tengan el campo \texttt{track}.
\end{quote}

\begin{quote}
\textbf{Pregunta 2:} El campo \texttt{track} contiene la información de
los puntos en los que el viaje ha sido trazado. Para aquellos documentos
que tienen en campo \texttt{track}, calcula el número de elementos del
\emph{subcampo} \texttt{features} de track.
\end{quote}

\begin{quote}
\textbf{Pregunta 3:} En \texttt{track.features} hay un array de
documentos con \texttt{geometry.time}. ¿Todos los documentos con campo
\texttt{track} tienen \texttt{track.features.geometry.time} con valor
``Point''?
\end{quote}

\begin{quote}
\textbf{Pregunta 4:} ¿Cuál es el tiempo de viaje máximo
(\texttt{travel\_time}), mínimo y medio que ha tardado en recorrer un
usuario que partía de la estación 14?
\end{quote}

\begin{quote}
\textbf{Pregunta 5:} Calcula la media de \texttt{travel\_time} para cada
día del mes. El resultado debe estar ordenado desde el día 1 del mes
hasta el último.
\end{quote}

\begin{quote}
\textbf{Pregunta 6:} ¿Sabrías decir cuántas estaciones están situadas en
la calle Segovia?
\end{quote}

\begin{quote}
\textbf{Pregunta 7:} El campo \texttt{\_id} está almacenado como una
cadena de caracteres (\emph{string}). Convierte el campo para que
aparezca como una fecha, es decir, del estilo
\texttt{ISODate("2018-12-01T00:30:12.524Z")}.
\end{quote}

\begin{quote}
\textbf{Pregunta 8:} ¿Cuál fue la estación o estaciones que más bases
libres (\texttt{free\_bases}) tuvieron el día 8 de diciembre?
\end{quote}

\begin{quote}
\textbf{Pregunta 9:} Calcula el tiempo medio de viaje entre cada par de
estaciones
\end{quote}

\begin{quote}
\textbf{Pregunta 10:} Genera un nueva colección denominada
\texttt{stations2} que contenga un documento para cada estación y cada
documento un array \texttt{states} con el par
\texttt{\{timestamp,\ free\_bases\}}. El resultado debería ser algo
similar a
\end{quote}

\begin{verbatim}
{
    "_id" : 136,
    "states" : [
        {
            "timestamp" : ISODate("2018-12-01T00:30:12.524Z"),
            "free_bases" : 19
        },
        {
            "timestamp" : ISODate("2018-12-01T01:30:12.684Z"),
            "free_bases" : 9
        },
        {
            "timestamp" : ISODate("2018-12-01T03:30:15.860Z"),
            "free_bases" : 0
        },
        {<más documentos>...}
    ]
}
\end{verbatim}


\end{document}
