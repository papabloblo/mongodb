\documentclass[]{article}
\usepackage{lmodern}
\usepackage{amssymb,amsmath}
\usepackage{ifxetex,ifluatex}
\usepackage{fixltx2e} % provides \textsubscript
\ifnum 0\ifxetex 1\fi\ifluatex 1\fi=0 % if pdftex
  \usepackage[T1]{fontenc}
  \usepackage[utf8]{inputenc}
\else % if luatex or xelatex
  \ifxetex
    \usepackage{mathspec}
  \else
    \usepackage{fontspec}
  \fi
  \defaultfontfeatures{Ligatures=TeX,Scale=MatchLowercase}
\fi
% use upquote if available, for straight quotes in verbatim environments
\IfFileExists{upquote.sty}{\usepackage{upquote}}{}
% use microtype if available
\IfFileExists{microtype.sty}{%
\usepackage{microtype}
\UseMicrotypeSet[protrusion]{basicmath} % disable protrusion for tt fonts
}{}
\usepackage[margin=1in]{geometry}
\usepackage{hyperref}
\hypersetup{unicode=true,
            pdftitle={MongoDB},
            pdfauthor={Pablo Hidalgo},
            pdfborder={0 0 0},
            breaklinks=true}
\urlstyle{same}  % don't use monospace font for urls
\usepackage{graphicx,grffile}
\makeatletter
\def\maxwidth{\ifdim\Gin@nat@width>\linewidth\linewidth\else\Gin@nat@width\fi}
\def\maxheight{\ifdim\Gin@nat@height>\textheight\textheight\else\Gin@nat@height\fi}
\makeatother
% Scale images if necessary, so that they will not overflow the page
% margins by default, and it is still possible to overwrite the defaults
% using explicit options in \includegraphics[width, height, ...]{}
\setkeys{Gin}{width=\maxwidth,height=\maxheight,keepaspectratio}
\IfFileExists{parskip.sty}{%
\usepackage{parskip}
}{% else
\setlength{\parindent}{0pt}
\setlength{\parskip}{6pt plus 2pt minus 1pt}
}
\setlength{\emergencystretch}{3em}  % prevent overfull lines
\providecommand{\tightlist}{%
  \setlength{\itemsep}{0pt}\setlength{\parskip}{0pt}}
\setcounter{secnumdepth}{0}
% Redefines (sub)paragraphs to behave more like sections
\ifx\paragraph\undefined\else
\let\oldparagraph\paragraph
\renewcommand{\paragraph}[1]{\oldparagraph{#1}\mbox{}}
\fi
\ifx\subparagraph\undefined\else
\let\oldsubparagraph\subparagraph
\renewcommand{\subparagraph}[1]{\oldsubparagraph{#1}\mbox{}}
\fi

%%% Use protect on footnotes to avoid problems with footnotes in titles
\let\rmarkdownfootnote\footnote%
\def\footnote{\protect\rmarkdownfootnote}

%%% Change title format to be more compact
\usepackage{titling}

% Create subtitle command for use in maketitle
\providecommand{\subtitle}[1]{
  \posttitle{
    \begin{center}\large#1\end{center}
    }
}

\setlength{\droptitle}{-2em}

  \title{MongoDB}
    \pretitle{\vspace{\droptitle}\centering\huge}
  \posttitle{\par}
  \subtitle{Taller de NoSQL}
  \author{Pablo Hidalgo}
    \preauthor{\centering\large\emph}
  \postauthor{\par}
    \date{}
    \predate{}\postdate{}
  

\begin{document}
\maketitle

En esta práctica puedes practicar todo lo que hemos visto de MongoDB.
Cualquier duda que tengas, puede ser útil tener a mano la
\href{https://docs.mongodb.com/manual/}{documentación de MongoDB}.

Vamos a comenzar insertando los siguientes documentos en una colección
llamada \texttt{movies}:

\begin{verbatim}
title : "Fight Club",
writer : "Chuck Palahniuk",
year : 1999,
actors : [
  "Brad Pitt",
  "Edward Norton"
]
\end{verbatim}

\begin{verbatim}
title : "Pulp Fiction",
writer : "Quentin Tarantino",
year : 1994,
actors : [
  "John Travolta",
  "Uma Thurman"
]
\end{verbatim}

\begin{verbatim}
title : "Inglorious Basterds",
writer : "Quentin Tarantino",
year : 2009,
actors : [
  "Brad Pitt",
  "Diane Kruger",
  "Eli Roth"
]
\end{verbatim}

\begin{verbatim}
title : "The Hobbit: An Unexpected Journey",
writer : "J.R.R. Tolkein",
year : 2012,
franchise : "The Hobbit"
\end{verbatim}

\begin{verbatim}
title : "The Hobbit: The Desolation of Smaug",
writer : "J.R.R. Tolkein",
year : 2013,
franchise : "The Hobbit"
\end{verbatim}

\begin{verbatim}
title : "The Hobbit: The Battle of the Five Armies",
writer : "J.R.R. Tolkein",
year : 2012,
franchise : "The Hobbit",
synopsis : "Bilbo and Company are forced to engage in a war against an array of combatants and keep the Lonely Mountain from falling into the hands of a rising darkness."
\end{verbatim}

\begin{verbatim}
title : "Pee Wee Herman's Big Adventure"
\end{verbatim}

\begin{verbatim}
title : "Avatar"
\end{verbatim}

\hypertarget{encontrando-documentos}{%
\section{Encontrando documentos}\label{encontrando-documentos}}

\begin{enumerate}
\def\labelenumi{\arabic{enumi}.}
\tightlist
\item
  Devuelve todas las películas.
\item
  Extrae todas las películas cuyo guionista es Quentin Tarantino.
\item
  Extrae aquellas películas donde actúe Brad Pitt.
\item
  Extrae aquellas películas rodadas en los años 90.
\item
  Extrae aquellas películas rodadas antes del año 2000 o después del
  2010.
\end{enumerate}

\hypertarget{actualizando-documentos}{%
\section{Actualizando documentos}\label{actualizando-documentos}}

\begin{enumerate}
\def\labelenumi{\arabic{enumi}.}
\item
  Actualiza la sinposis de
  \texttt{The\ Hobbit:\ An\ Unexpected\ Journey} a
  \texttt{A\ reluctant\ hobbit,\ Bilbo\ Baggins,\ sets\ out\ to\ the\ Lonely\ Mountain\ with\ a\ spirited\ group\ of\ dwarves\ to\ reclaim\ their\ mountain\ home\ -\ and\ the\ gold\ within\ it\ -\ from\ the\ dragon\ Smaug.}
\item
  Añade el actor \texttt{Samuel\ L.\ Jackson} a la película
  \texttt{Pulp\ Fiction}.
\end{enumerate}

\hypertarget{eliminando-documentos}{%
\section{Eliminando documentos}\label{eliminando-documentos}}

\begin{enumerate}
\def\labelenumi{\arabic{enumi}.}
\tightlist
\item
  Elimina la película
  \texttt{Pee\ Wee\ Herman\textquotesingle{}s\ Big\ Adventure}.
\item
  Elimina la película \texttt{Avatar}.
\end{enumerate}


\end{document}
